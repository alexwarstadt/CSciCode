%%%%%%%%%%%%%%%%%%%%%%%%%%%%%%%%%%%%%%%%%%%%%%%%%%%%%%%%%%%%%%%%%%%%%%%%%%%%%%%%%%%%%%%%%%%%%%%%%%%%%
%Include files, leave this alone
\title{CS18 HW7.2 analysis}
\documentclass[12pt,letterpaper]{article}
\usepackage{amsmath}
\usepackage{fullpage}
\usepackage{enumerate}
\usepackage{fancyhdr}
\setlength{\parindent}{0.0in}
\setlength{\parskip}{0.1in}
\newcommand{\tab}{\hspace{3em}}
%%%%%%%%%%%%%%%%%%%%%%%%%%%%%%%%%%%%%%%%%%%%%%%%%%%%%%%%%%%%%%%%%%%%%%%%%%%%%%%%%%%%%%%%%%%%%%%%%%%%%


% Edit these as appropriate
\newcommand\course{CS 18}
\newcommand\semester{Spring 2014}       % <-- current semester
\newcommand\hwnum{7}                  % <-- homework number

%%%%%%%%%%%%%%%%%%%%%%%%%%%%%%%%%%%%%%%%%%%%%%%%%%%%%%%%%%%%%%%%%%%%%%%%%%%%%%%%%%%%%%%%%%%%%%%%%%%%%
%This section setups the header for the document, leave this alone
\pagestyle{fancy}
\headheight 28pt
\fancyhead[R]{\course, \semester\\ Homework \hwnum, \today}
%%%%%%%%%%%%%%%%%%%%%%%%%%%%%%%%%%%%%%%%%%%%%%%%%%%%%%%%%%%%%%%%%%%%%%%%%%%%%%%%%%%%%%%%%%%%%%%%%%%%%

\newenvironment{answer}[2]{
  \section*{Problem \hwnum.#1}
}{\newpage}

%%%%%%%%%%%%%%%%%%%%%%%%%%%%%%%%%%%%%%%%%%%%%%%%%%%%%%%%%%%%%%%%%%%%%%%%%%%%%%%%%%%%%%%%%%%%%%%%%%%%%
%The document begins here!
\begin{document}

\begin{answer}{2}

\textbf{Recurrence relation for Currency Exchange}\\
Let CE give the maximum amount of money in Wesels that can result from $n$ transactions starting with an initial amount of Wesels $a$. Let CE have access to a table of exchange rates with the dimensions $j \times j$. The value for $i$ passed into CE is the index of the row in the $rates$ table corresponding to the type of currency of the amount passed in. The Wesel row and column are both at index $i=0$.

\[ CE(n, a, i) = \left\{ 
  \begin{array}{l l}
    a & \quad \text{if $n = 0$}\\
    (a)rates(i)(0) & \quad \text{if $n = 1$}\\
    max(CE(n-1, (a)(rates(i)(0)), 0), \\
    \;\;\;\;\;\;\;\;CE(n-1, (a)(rates(i)(1)), 1), \dots,\\
    \;\;\;\;\;\;\;\;CE(n-1, (a)(rates(i)(j-1)), j-1)) & \quad \text{otherwise}
  \end{array} \right.\]\\


\par
\textbf{Analysis for naive Currency Exchange} \\
A naive implementation of the above recurrence would have a runtime on the order of $O(j^n)$, where, as mentioned above $j$ is the dimension of the $rates$ table (also the number of currencies) and $n$ the number of transactions. The reason for this is that at each recursive step of CE $j$ recursive calls are made, resulting in a $j$-ary branching tree of depth $n+1$. At each recursive step, only constant operations are performed.\\\\

\textbf{Analysis for DP Currency Exchange} \\
The DP solution has a runtime on the order of $O(j^2n)$. Aside from a few constant steps of constant runtime, all of the work is done building a two-dimensional array to store intermediate maximum values. This array has dimensions $(n+1) \times j$. At each of $n$ iterative steps to build this array, $j$ optimal values must be calculated, one for each type of currency. Each optimal value is calculated by taking the maximum value attained by converting the $j$ optimal values from the previous iteration. Thus, each of $n$ iterative step performs $j \times j$ operations.

\end{answer}


\end{document}