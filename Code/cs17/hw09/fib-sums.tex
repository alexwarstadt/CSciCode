%%%%%%%%%%%%%%%%%%%%%%%%%%%%%%%%%%%%%%%%%%%%%%%%%%%%%%%%%%%%%%%%%%%%%%%%%%%%%%%%%%%%%%%%%%%%%%%%%%%%%
%Include files, leave this alone
\documentclass[12pt,letterpaper]{article}
\usepackage{amsmath}
\usepackage{fullpage}
\usepackage{enumerate}
\usepackage{fancyhdr}
\setlength{\parindent}{0.0in}
\setlength{\parskip}{0.1in}
\newcommand{\tab}{\hspace{3em}}
%%%%%%%%%%%%%%%%%%%%%%%%%%%%%%%%%%%%%%%%%%%%%%%%%%%%%%%%%%%%%%%%%%%%%%%%%%%%%%%%%%%%%%%%%%%%%%%%%%%%%


% Edit these as appropriate
\newcommand\course{CS 17}
\newcommand\semester{Fall 2013}       % <-- current semester
\newcommand\hwnum{9}                  % <-- homework number

%%%%%%%%%%%%%%%%%%%%%%%%%%%%%%%%%%%%%%%%%%%%%%%%%%%%%%%%%%%%%%%%%%%%%%%%%%%%%%%%%%%%%%%%%%%%%%%%%%%%%
%This section setups the header for the document, leave this alone
\pagestyle{fancy}
\headheight 28pt
\fancyhead[R]{\course, \semester\\ Homework \hwnum - \today}
%%%%%%%%%%%%%%%%%%%%%%%%%%%%%%%%%%%%%%%%%%%%%%%%%%%%%%%%%%%%%%%%%%%%%%%%%%%%%%%%%%%%%%%%%%%%%%%%%%%%%

\newenvironment{answer}[5]{
  \section*{Problem \hwnum.#1}
}{\newpage}

%%%%%%%%%%%%%%%%%%%%%%%%%%%%%%%%%%%%%%%%%%%%%%%%%%%%%%%%%%%%%%%%%%%%%%%%%%%%%%%%%%%%%%%%%%%%%%%%%%%%%
%The document begins here!
\begin{document}

\begin{answer}{5}
\textbf{Analysis 1} \\

Task 1\\
The recurrence relation G(n) is as follows:
\[ G(n) = \left\{ 
  \begin{array}{l l}
    1 & \quad \text{if $n = 0$}\\
    F(n) + G(n - 1) & \quad \text{if $n > 1$}
  \end{array} \right.\]\\
  
Task 2\\

Pf:\\
  Basis: In the base case, n = 1,
  \begin{align*}
G(1) &= 1 \\
&= 2 - 1 \\
&= F(3) - 1 \\
&= F(1+2) - 1
\end{align*}

  Step: Assume
  \begin{align*}
G(n - 1) &= F((n - 1) + 2) -1 \\
&= F(n + 1) -1\\
\end{align*}

  Then
  \begin{align*}
G(n) &= F(n) + G(n - 1) \\
&= F(n) + F(n + 1) -1\\
&= F(n + 2) - 1
\end{align*}
\begin{flushright} by the recurrence relation F(n) \end{flushright}
Q.E.D.

\end{answer}


\end{document}